\documentclass[a4paper, fontsize=10pt, oneside]{article}
\usepackage[T2A]{fontenc}
\usepackage[utf8x]{inputenc}    
\usepackage[ukrainian, english]{babel} 
\usepackage{setspace}
\usepackage{misccorr} 
\usepackage{ragged2e}

\onehalfspacing
\justifying

\begin{document}
\begin{table}
\caption{Розрахунок вирівених значень $T$ і похибок $\epsilon$. Таблиця...}
\begin{tabular}{|c|c|c|c|c|c|c|c|c|}
\hline
$t$ & $y_{t}$ & $S_{t}$ & $T \cdot \epsilon = y : S$ & $T$ & $T \cdot S$ & $\epsilon = y : (T \cdot S)$ & $\epsilon^{1} = y - (T \cdot S)$ & $(\epsilon^{1})^2$\\
1 & 2 & 3 & 4 & 5 & 6 & 7 & 8 & 9\\
1 & 72 & 0.913 & 78.86 & 87.80 & 80.16 & 0.898 & -8.16 & 66.66\\
\hline
2 & 100 & 1.202 & 83.19 & 85.03 & 102.20 & 0.978 & -2.20 & 4.86\\
\hline
3 & 90 & 1.082 & 83.18 & 82.25 & 89.00 & 1.011 & 1.00 & 1.00\\
\hline
4 & 64 & 0.803 & 79.70 & 79.48 & 63.82 & 1.003 & 0.18 & 0.03\\
\hline
5 & 70 & 0.913 & 76.67 & 76.70 & 70.03 & 1.00 & -0.03 & 0.00\\
\hline
6 & 92 & 1.202 & 76.54 & 73.93 & 88.86 & 1.035 & 3.14 & 9.85\\
\hline
7 & 80 & 1.082 & 73.94 & 71.15 & 16.99 & 1.039 & 3.01 & 9.08\\
\hline
8 & 58 & 0.803 & 72.23 & 68.38 & 4.91 & 1.056 & 3.09 & 9.57\\
\hline
9 & 62 & 0.913 & 67.91 & 65.60 & 59.90 & 1.035 & 2.10 & 4.43\\
\hline
10 & 80 & 1.202 & 66.56 & 62.83 & 75.52 & 1.059 & 4.48 & 20.08 \\
\hline
11 & 68 & 1.082 & 62.85 & 60.05 & 64.98 & 1.047 & 3.02 & 9.14\\
\hline
12 & 48 & 0.803 & 59.78 & 57.28 & 45.99 & 1.044 & 2.01 & 4.03\\
\hline
13 & 52 & 0.913 & 56.96 & 54.50 & 49.76 & 1.045 & 2.24 & 5.02\\
\hline
14 & 60 & 1.202 & 49.92 & 51.73 & 62.18 & 0.965 & -2.18 & 4.73\\
\hline
15 & 50 & 1.082 & 46.21 & 48.95 & 52.97 & 0.944 & -2.97 & 8.79\\
\hline
16 & 30 & 0.803 & 37.36 & 46.18 & 37.08 & 0.809 & -7.08 & 50.12\\
\hline
\end{tabular}
\end{table}

\addtocounter{section}{3}

\section{\rm \normalsize Розрахуємо компоненту $T$. Для цього розрахуємо параметри лінійного тренду, використовуючи рівні $(T \cdot \epsilon)$/.}
Метод найменших квадратів дає таку модель:
$$T = 90.59 - 2.773t.$$
В отримане рівняння підкладем $t = 	\overline{1.16}$ і отримаємо рівні $T$ для всіх моментів часу (графа 5).
\section{\rm \normalsize Розрахуємо значення $T \cdot S$ (графа 6)}
\section{\rm \normalsize Далі визначаємо похибку в мультиплікативній моделі: $\epsilon = y : (T \cdot S)$ (графа 7)}
Для порівняння мультиплікативної моделі з іншими моделями часового ряду, по анології з адитивною моделлю використовують суму квадратів абсолютних похибок.

\chapter{\Large \bf Застосування фіктивних змінних для моделювання сезонних коливань}
Моделювати часовий ряд з сезонними коливаннями можна за допомогою побудови регресії з включенням фактора часу і фіктивних змінних. Кількість фіктивних змінних в такій моделі повинна бути на одиницю меньше кількості періодів часу всередині одного циклу коливань. Наприклад, якщо моделюємо поквартальні дані, то модель включатиме фактор часу і три фіктивні змінні. Кожна фіктивна змінна відображає сезонну (циклічну) компонентучасового ряду для якогось одного періоду. Вона дорівнює одиниці для одного періоду і нулеві для інших.

Якщо розглядається часовий рід з циклічними коливаннями з періодичністю $K$, то модель з фіктивними змінними матиме вигляд:

$$y_{t} = a +bt + c_{1}x_{1} + \cdots +c_{k-1}x_{k-1} + \epsilon_{t},$$

\noindent де $x_{j} = 
\left\{
  \begin{array}{l}
     1, \texttt{ для кожного } j \texttt{ всередині кожного циклу}\\
     0, \texttt{ в інших випадках} \\
  \end{array}
\right.$

Наприклад, якщо моделювати сезонні коливання на основі поквартальних даних за декілька років,кількість кварталів в одному році $k = 4$, а модель матиме вигляд: 

$$y_{t} = a + bt + c_{1}x_{1} + c_{2}x_{2} + c_{3}x_{3} + \epsilon_{t}$$

Тоді рівняння тренду для кожного квартала матиме такий вигляд:

\begin{tabular}{l l}
для 1 квартала: & $y_{t} = a + bt + c_{1} + \epsilon_{t}$\\
для 2 квартала: & $y_{t} = a + bt + c_{2} + \epsilon_{t}$\\
для 3 квартала: & $y_{t} = a + bt + c_{3} + \epsilon_{t}$\\
для 4 квартала: & $y_{t} = a + bt + \epsilon_{t}$\\
\end{tabular}

Це рівняння є аналог  адитивної моделі часового  ряду, так як фактичний рівень часового ряду є сума трендової, сезонної і випадкової компоненти.

\subsubsection*{Приклад} Побудова моделі регресії часового ряду з фіктивними змінними.

По даним прикладу про споживання електроенергії побудуємо модель регресії з фіктивними змінними.

Модель матиме вигляд: $y_{t} = a + bt + c_{1}x_{1} + c_{2}x_{2} + c_{3}x_{3} + \epsilon_{t}$

Матриця початкових даних матиме вигляд:


\begin{tabular}{|c|c|c|c|c|}
\hline
$y$ & $t$ & $x_{1}$ & $x_{2}$ & $x_{3}$\\
\hline
6.0 & 1 & 1 & 0 & 0\\
4.4 & 2 & 0 & 1 & 0\\
5.0 & 3 & 0 & 0 & 1\\
9.0 & 4 & 0 & 0 & 0\\
7.2 & 5 & 1 & 0 & 0\\
4.8 & 6 & 0 & 1 & 0\\
6.0 & 7 & 0 & 0 & 1\\
10.0 & 8 & 0 & 0 & 0\\
8.0 & 9 & 1 & 0 & 0\\
5.6 & 10 & 0 & 1 & 0\\
6.4 & 11 & 0 & 0 & 1\\
11.0 & 12 & 0 & 0 & 0\\
9.0 & 13 & 1 & 0 & 0\\
6.6 & 14 & 0 & 1 & 0\\
7.0 & 15 & 0 & 0 & 1\\
10.8 & 16 & 0 & 0 & 0\\
\hline
\end{tabular}


Методом найменьших квадратів отримаємо таку регресію: $\hat{y}_{t} = 8.3250 + 0.1875t - 2.0875x_{1} - 404750x_{2} - 3.9125x_{3}$

\subsection*{Проаналізуємо отримані результати}

Коєфіцієнт детермінації моделі $R^2 = 0.985$ досить високий, близький доодиниці. Параметр $a = 8.3250$ є сумапочаткового рівня ряду і  сезонної компоненти в \MakeUppercase{\romannumeral 4} кварталі (так як для четвертого кварталу $x_{1} = x_{2} = x_{3} = 0$). Сезонні коливання в \MakeUppercase{\romannumeral 1, \romannumeral 2, \romannumeral 3} кварталах зменьшують цю величину (від'ємні знаки при  $x_{1}, x_{2}, x_{3}$). Зазначимо, що ці параметри при $x_{i}$ не дорівнюють значенням сезонної компоненти, так як вони характеризують не сезонні зміни рівнів ряду, а їх відхилення від рівнів, що враховують сезонні впливи в \MakeUppercase{\romannumeral 4} кварталі. Додатна величина $b = 0.1875$ вказує про присутність зростаючого тренду.

Основний недолік моделіз фіктивними змінними той, що до моделі може вводитися велика кількість фіктивних зміних, а тому довжину ряду необхідно брати досить великою.

\chapter{\Large \bf Моделювання тенденції часового ряду при наявності структурних змін}

Розглянемо випадок лінійної тенденції. Існують одночасні зміни характеру тенденції часового ряду, що викликані структурними змінами в аналізуємому процесі. Тобто, починаючи з деякого моменту часу я $t^{*}$ відбувається зміна характеру динаміки ряду.

У момент $t^{*}$ відбуваються значні зміни факторів, зо входять до часового ряду. Тому виникає питання, чи значимо впливають зміни факторів на характер тенденції.

Якщо вплив цей значимий, то для моделювання тенденції даного часового ряду необхідно використати кусочно-??? моделі регресії. Це означає, що ряд розбиваємо на дві частини: одна до моменту $t^{*}$, а друга - після.

???

\end{document}