
\documentclass[a4paper, fontsize=10pt, oneside]{article}
\usepackage[T2A]{fontenc}
\usepackage[utf8x]{inputenc}    
\usepackage[ukrainian, english]{babel} 
\usepackage{setspace}
\usepackage{misccorr} 
\usepackage{ragged2e}

\onehalfspacing
\justifying

\begin{document}

\bf Приклад. \rm Є данні про споживання електроенергії жителями міста за дев'ять місяців
\newline  \newline 
\begin{tabular}{| c | c | c |}
\hline
місяць &об'єм споживання електроенергії жителями міста &знаки відхилень \\
\hline
січень &4.5 &- \\
\hline
лютий &5.2 &+ \\
\hline
березень &5.3 &+ \\
\hline
квітень &6.7 &+ \\
\hline
травень &6.1 &- \\
\hline
червень &6.4 &+ \\
\hline
липень &5.8 &- \\
\hline
серпень &5.0 &- \\
\hline
вересень &5.3 &- \\
\hline
\end{tabular}
\newline  \newline 
Далі будуємо по отриманим данним знаків відхилення ряд їх розподілу:
\newline \newline 
\begin{tabular}{| c | c | c |}
\hline
Вид тенденції &Довжина сприятливої тенденціїї, $d_{i}$ &Частота, $c_{i}$ \\
\hline
- - & 0 &2 \\
\hline
- + -& 1 &1 \\
\hline
- + + - & 2 &0 \\
\hline
- + + + - & 3 &1 \\
\hline
\end{tabular}
\newline  \newline 

Розрахуємо середню довжину сприятливої тенденції
$$\overline{d} = \frac{\sum c_{i}d_{i}}{\sum C_{i}} = \frac{0\cdot2 + 1\cdot1 + 2\cdot0 + 3 \cdot 1}{2 + 1 + 0 +1}.$$

Інтенсивність переривання сриятливої тенденциї $(\lambda)$ така: $$\lambda = \frac{1}{d} = \frac{1}{1} = 1.$$

Ймовірність спостереження сприятливої тенденції буде такою:
\newline \newline
\begin{tabular}{|c|c|c|c|c|}
\hline
Період збереження сприятливої тенденції & $t$ & $\lambda$ & $-\lambda t$ & ймовірність сприятливої тенденциї, $e^{-\lambda t}$ \\
\hline
жовтень & 1 & 1 & -1 & 0.368\\
\hline
листопад & 2 & 1 & -2 & 0.135\\
\hline
грудень & 3 & 1 & -3 & 0.049\\
\hline
\end{tabular}
\newline \newline
Отримали, що ймовірність зростання споживання електроенергії жителями міста в жовтні в порівнянні з вереснем дорівнює 0.368

\begin{center}
\chapter{\Large \bf Оцінка точності прогнозу}
\end{center}

Оцінка точності прогнозу є важливим етапом прогнозування. Для  оцінки точності прогнозу використовується різниця між прогнозним значенням $\hat{y}^{*}_{t}$ і фактичним $y_{t}$ значенням показника. Цей підхід можна  використовувати в таких випадках:
\begin{itemize}
\item період випередження відомий і відомі фактичні значення прогнозного показника;
\item будується ретроспективний прогноз, тобто розраховуються прогнозні значення показника для періоду часу, для якого є фактичні значення.
\end{itemize}
В цьому випадку інформация ділиться на дві частини в співвідношенні 2/3 до 1/3. Перша частина значень рівнів використовується для визначення параметрів моделі прогнозу. Друга частина інформації служить для розрахунку оцінок прогнозу.

Розглянемо деякі показники точності прогнозу:
\section{Абсолютна похибка прогнозу}
Вона визначається якрізниця між емпіричними і прогнозними значеннями показника за формулою:
$$\Delta a = y_{t} - \hat{y_{t}^{*}},$$
\begin{tabular}{l l}
де & $y_{t}$ - фактичні значення показника,\\
& $\hat{y}_{t}^{*}$ - прогнозні значенния показника.\\
\end{tabular} 
\section{Відносна похибка прогнозу}
Вона розраховується двома способами:
$$\Delta b = \frac{\Delta a}{y_{t}} = \frac{(y_{t} - \hat{y}_{t}^{*})}{y_{t}} \cdot 100\% \texttt{ і}$$

$$\Delta b = \frac{\Delta }{\hat{y}_{t}} = \frac{(y_{t} - y_{t}^{*})}{\hat{y}_{t}} \cdot 100\% $$

Зазначимо, щоабсолютна і відносна похибки прогнозу є оцінкою точності одиничного прогнозу, що не дає можливості говорити про їх важливість в оцінці всієї прогнозної моделі.
\section{\rm \normalsize Тому на практиці інколи визначають не похибку прогнозу, а коефіцієнт якості прогнозу за формулою:}
$$ K = \frac{C}{C + H}\texttt{, де}$$

$C$ - кількість прогнозів, співпавших з фактичними значеннями,

$H$ - не співпавших.

Якщо $K = 1$, то це означає, що всі значення прогнозних і вактичних значень співпадають і модель на 100\% описує явище.
\section{Середній показник точності прогнозу}
Цей показник розраховується так:
$$\overline{\Delta} = \frac{\sum \limits_{i = 1}^{n} \Delta t}{n} =  \frac{\sum \limits_{t = 1}^{n} |y_{t} - \hat{y}_{t}|}{n},$$
\noindent де $n$ - довжина частини або всього ряду, на якому зрівнюються прогнозні і фактичні рівні. Показник показує узагальнену характеристику відхилень фактичних і прогнозних значень показника.
\section{Середня квадратична похибка прогнозу}
Вона розраховується таким чином:

$$ \delta = \sqrt{\frac{\sum\limits_{t=1}^{n}(y_{t} - \hat{y}_{t}^{t})^2}{n}} $$

Між середньою абсолютною і середньою квадратичною похибками прогнозу інує таке співвідношення:
$$ \delta = 1.25 \overline{\Delta}.$$
Недоліком двох останніх параметрів є їхня суттєва залежність від масштабу виміру рівнів.
\section{Середня похибка впроксимації}
Щоб звільнитися від масштабу вікористовують середню похибку апроксимації
$$ MAPE = \frac{1}{n} \sum_{t=1}^{n} \frac{(y_{t} - \hat{y}_{t}^{t})}{y_{t}} 100\%.$$
Інтерпритація оцінки точності прогнозу за цим показником представлена в таблиці:

\begin{tabular}{|l|l|}
\hline
MAPE &Інтерпритація точності\\
\hline
$<10$ & висока\\
$10 \div 20$ & добра\\
$20 \div 50$ & задовільна\\
$>10$ & незадовільна\\
\hline
\end{tabular}

\section{Коефіціент невідповідності}

Цей показник був запропонований Г.Тейлом і має декілька модифікацій:
$$KH_{1} = \sqrt{\frac{\sum (\hat{y}^{*}_{t} - y_{t})^2}{\sum\limits_{t=1}^{n} y_{t}^2}}, KH_{2} = \sqrt{\frac{\sum\limits_{t=1}^{n}(\hat{y}^{*}_{t} - y_{t})^2}{\sum\limits_{t=1}^{n}(y_{t} - \overline{y}_{t})^2}},$$
\noindent де $\overline{y}$ - середній рівень елементів ряду.
\end{document}