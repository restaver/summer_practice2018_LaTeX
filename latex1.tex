\documentclass[11pt]{article}
%криллица
\usepackage[utf8]{inputenc}
\usepackage[english, russian]{babel}
%эпиграф
\usepackage{epigraph}
\setlength{\epigraphwidth}{0.64\textwidth}
\setlength{\epigraphrule}{0pt}

\usepackage{ragged2e} % justifying

 \usepackage[left=2.5cm,right=2cm,bindingoffset=0cm]{geometry}%поля


\begin{document}
    \centering \Huge \bf{Сложность}
    \rm
    \epigraph {\justifying     \uchyph=0%запрет перенососв слов
    Врач, строитель и программистка спорили о том, чья профессия древнее. Врач заметил: ''В Библии сказано, что Бог сотворил Еву из ребра Адама. Такая операция может быть проведена только хирургом, поэтому я по праву могу утверждать, что моя профессия самая древняя в мире''. Тут вмешался строитель и сказал: ''Но еще раньше в Книге Бытия сказано, что Бог сотворил из хаоса небо и землю. Это было первое и, несомненно, наиболее выдающееся строительство. Поэтому, дорогой доктор, вы не правы. Моя профессия самая древняя в мире''. Программистка при этих словах откинулась в кресле и с улыбкой произнесла: ''А кто же по-вашему сотворил хаос?''}
    
    \section{Cложность, присущая программному обеспечению }
    	\subsection{Простые и сложные программные системы }
    	\subsection{Почему программному обеспечению присуща сложность? }
    	\subsection{Последствия неограниченной сложности }
    \section{Структура сложных систем}
    	\subsection{Примеры сложных систем }
    	\subsection{Пять признаков сложной системы }
    \section*{Список литературы}
    
    
\end{document}