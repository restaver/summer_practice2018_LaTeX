
\documentclass[a4paper, fontsize=10pt, oneside]{article}
\usepackage[T2A]{fontenc}
\usepackage[utf8x]{inputenc}    
\usepackage[ukrainian, english]{babel} 
\usepackage{setspace}
\usepackage{misccorr} 
\usepackage{ragged2e}

\onehalfspacing
\justifying

\begin{document}

\chapter{\Large \bf 4.Проста експонента} 

Розглянемо тенденцію виду: $y = ab^t$. Логарифмуемо це співвідношення:

$$ \ln{y_{t}} = \ln{a} + t\ln{b}$$

Отримали лінійну функцію відносно $t$. Параметри $a$ і $b$ розраховуємо так, як і у випадку лінійного тренду. Для спрощеного розрахунку інтервалу можна скористатися параметром $k^{*}$ для лінійного тренду.

Тоді прогнозний інтервал матиме вигляд:
$$H.M.:ant \, ln{(\ln{\hat{y}_{n+L}-S_{y}k^{*}})},$$
$$B.M.:ant \, ln{(\ln{\hat{y}_{n+L}+S_{y}k^{*}})},$$
\noindent де $S_{y} = \sqrt{\frac{(\ln{y_{t}} - \ln{\hat{y}_{t}})^2}{n-2}}$

\chapter{\Large \bf 5.Логарифмична парабола}

Розглядається тенденція виду $y_{t} = a\,b^{t}\,c^{t^2}$. Логарифмуємо цей вираз: $\ln{y_{t}} = \ln{a} + t\,\ln{b} + t^2\,\ln{c}$. Це квадратична функція відносно параметра $t$. Оцінку параметрів $a, b, c$ і прогнозний інтервал знаходимо так, як для многочлена другого ступеня.

Прогнозний інтервал буде таким:

$$H.M.:ant \, ln{(\ln{\hat{y}_{n+L}-S_{y}k^{*}})},$$
$$B.M.:ant \, ln{(\ln{\hat{y}_{n+L}-S_{y}k^{*}})},$$

\noindent де $k^{*}$ - табульовані значення для многочлена другого ступеня, а $S_{y} = \sqrt{\frac{(\ln{y_{t}} - \ln{\hat{y}_{t}})^2}{n-3}}.$

\chapter{\Large \bf 6.Модифікована експонента}

Тенднція має вигляд: $y_{t} = c - a\,b^{t}.$ Вважаємо, що параметр $a>0.$ Вираз зведемо до лінійного виду: $\ln{(c - y_{t})} = \ln{ a} + t\,\ln{b}.$ Позначемо $z_{t} = \ln{(c^{*} - y_{t})}.$ Вважаємо, що $c = c^{*}$ - відоме, тобто, відома асимптота. Отримаємо такий вираз для прогнозного інтервалу:

$$H.M.:z_{n+L} - S_{z}\,k^{*},$$
$$H.M.:z_{n+L} + S_{z}\,k^{*},$$

\noindent де $S_{z}$ - середньоквадратичне віждхилення від тренду $z_{t} = 	\ln{ a} + t\, \ln{b}.$
Тоді прогнозний інтервал для $y_{n+L}$ має вигляд:

$$H.M.: c^{*} - ant\,ln{(z_{n+L} - S_{z} \, k^{*})},$$
$$B.M.: c^{*} - ant\,ln{(z_{n+L} + S_{z} \, k^{*})},$$

Якщо параметр $c$ невідомий, то можна оцінити параметри модифікованної експоненти $y = c + a \, b^{t}$ методом трьох сум. Згідно з цим методом часовий ряд розбиваємо на три однакових відрізка і позначимо через $\sum_{1}y_{t}, \sum_{2}y_{t}, \sum_{3}y_{t}$ суми рівнів кожного з відрізків, $n$ -кількість рівнів у кожному з відрізків. Вікростовуючи алгоритм методу трьох сум отримаємо такі оцінки параметрів $c$, $a$ і $b$:

$$ b = \sqrt[n]{\frac{\sum_{3}y_{t} - \sum_{2}y_{t}}{\sum_{2}y_{t} - \sum_{1}y_{t}}},\; a = ( \sum\nolimits_{2} y_{t} - \sum\nolimits_{1} y_{t} ) \frac{b - q}{(b^{n} - 1)^2},$$
$$ c = \frac{1}{n} \left[ \frac{\sum_{1}y_{t} \cdot \sum_{3}y_{t} - (\sum_{2}y_{t})^2}{\sum_{1} y_{t} + \sum_{3} y_{t} - 2 \sum_{2}y_{t}} \right].$$

Зауважимо, що метод трьох сум працює, якщо коливанняряду досить малі і результати не дуже чутливі до похибок. Тому перед оцінкою ряд необхідно згладити за допомогою ковзної середньої, якщо у ряді досить сильні коливання, або усунути досить великі викиди і замінити їх на усереднені.

\chapter{\Large \bf 7.Крива Гомперця}

Тенденція, що описується кривою Гомперця має вигляд $y = c \, a^{b^{t}}.$ За допомогою логарифмування криву Гомперця можна представити у вигляді модифікованної експоненти:

$$\ln{y_{t}} = \ln{c} + b^{t} \ln{a}$$

Застосувавши метод трьох сум отримаємо такі оцінки параметрів:

$$b = \sqrt[n]{\frac{\sum_{3}\ln{y_{t}} - \sum_{2}\ln{y_{t}}}{\sum_{2}\ln{y_{t}} - \sum_{1}\ln{y_{t}}}}, \; \ln{a} = \left(\sum\nolimits_{2}\ln{y_{t}} - \sum\nolimits_{1}\ln{y_{t}}\right)\frac{b-1}{(b^{n} - 1)^2},$$

$$ c = \frac{1}{n} \left[ \frac{\sum_{1}\ln{y_{t}} \cdot \sum_{3}\ln{y_{t}} - (\sum_{2}\ln{y_{t}})^2}{\sum_{1} \ln{y_{t}} + \sum_{3} \ln{y_{t}} - 2 \sum_{2}\ln{y_{t}}} \right].$$

Слід пам'ятати про зауваження, що робится при розрахунку параметрів модифікованної, відносно коливань і похибок.

\chapter{\Large \bf 8.Логістична крива Перла-Ріда}

\begin{itemize}

\item[a)] Якщо логістична крива має вигляд $$ \frac{1}{y_{t} = k + a\,b^{t}},$$ то застосувавшиметод трьох сум отримаємо такі оцінки параметрів $$ b = \sqrt[n]{\frac{\sum_{3}\frac{1}{y_{t}} - \sum_{2}\frac{1}{y_{t}}}{\sum_{2}\frac{1}{y_{t}} - \sum_{1}\frac{1}{y_{t}}}},\; a = ( \sum\nolimits_{2} \frac{1}{y_{t}} - \sum\nolimits_{1} \frac{1}{y_{t}} ) \frac{b - 1}{(b^{n} - 1)^2},$$ $$k = \frac{1}{n} \left( \sum\nolimits_{1}\frac{1}{y_{t}} - \frac{b^{n} - 1}{b - 1}a \right) \texttt{ або } k = \frac{1}{n} \left[ \frac{\sum_{1}\frac{1}{y_{t}} \cdot \sum_{3}\frac{1}{y_{t}} - (\sum_{2}\frac{1}{y_{t}})^2}{\sum_{1} \frac{1}{y_{t}} + \sum_{3} \frac{1}{y_{t}} - 2 \sum_{2}\frac{1}{y_{t}}} \right].$$

\item[б)] Нехай логістична крива представлена у вигляді
$$y_{t} = \frac{k}{1 + b e^{-at}}$$
Метод трьох сум дає такі оцінки параметрам логістичної кривої:
$$ a = \frac{1}{n} \left( \ln{D_{1}} -  \ln{D_{2}} \right), k = n : \left( \sum\nolimits_{1} \frac{1}{y_{t}} - \frac{D_{1}^{2}}{D_{1} - D_{2}} \right),$$
$$b = \frac{k}{c}\frac{D_{1}^{2}}{D_{1} - D_{2}}, \texttt{ де } c = \frac{1 - e^{-na}}{1 - e^{a}}, D_{1} = \sum\nolimits_{1}\frac{1}{y_{t}} - \sum\nolimits_{2}\frac{1}{y_{t}},$$
$$D_{2} = \sum\nolimits_{2}\frac{1}{y_{t}} - \sum\nolimits_{3}\frac{1}{y_{t}}.$$

\item[в)] Якщо логістична крива має вигляд
$$y_{t} = \frac{k}{1 + 10^{a + bt}}$$ 
і відсутній повний ряд даних, то для оцінки

\end{itemize}

\end{document}