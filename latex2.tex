
\documentclass[a4paper, fontsize=10pt, oneside]{article}
\usepackage[T2A]{fontenc}
\usepackage[utf8x]{inputenc}    
\usepackage[ukrainian, english]{babel} 
\usepackage{setspace}
\usepackage{misccorr} 

\onehalfspacing

\begin{document}

\chapter{\Large \bf 4.Проста експонента} 

Розглянемо тенденцію виду: $y = ab^t$. Логарифмуемо це співвідношення:

$$ \ln{y_{t}} = \ln{a} + t\ln{b}$$

Отримали лінійну функцію відносно $t$. Параметри $a$ і $b$ розраховуємо так, як і у випадку лінійного тренду. Для спрощеного розрахунку інтервалу можна скористатися параметром $k^{*}$ для лінійного тренду.

Тоді прогнозний інтервал матиме вигляд:
$$H.M.:ant \, ln{(ln{\,\hat{y}_{n+L}-S_{y}k^{*}})},$$
$$B.M.:ant \, ln{(ln{\,\hat{y}_{n+L}+S_{y}k^{*}})},$$
\noindent де $S_{y} = \sqrt{\frac{(ln{\,y_{t}} - ln{\,\hat{y}_{t}})^2}{n-2}}$

\chapter{\Large \bf 5.Логарифмична парабола}

Розглядається тенденція виду $y_{t} = a\,b^{t}\,c^{t^2}$. Логарифмуємо цей вираз: $ln{\,y_{t}} = ln{\,a} + t\,ln{\,b} + t^2\,ln{\,c}$. Це квадратична функція відносно параметра $t$. Оцінку параметрів $a, b, c$ і прогнозний інтервал знаходимо так, як для многочлена другого ступеня.

Прогнозний інтервал буде таким:

$$H.M.:ant \, ln{(ln{\,\hat{y}_{n+L}-S_{y}k^{*}})},$$
$$B.M.:ant \, ln{(ln{\,\hat{y}_{n+L}-S_{y}k^{*}})},$$

\noindent де $k^{*}$ - табульовані значення для многочлена другого ступеня, а $S_{y} = \sqrt{\frac{(ln{\,y_{t}} - ln{\,\hat{y}_{t}})^2}{n-3}}.$

\chapter{\Large \bf 6.Модифікована експонента}

Тенднція має вигляд: $y_{t} = c - a\,b^{t}.$ Вважаємо, ща параметр $a>0.$ Вираз зведемо до лінійного виду: $ln{\,(c - y_{t})} = ln{\, a} + t\,ln{\,b}.$ Позначемо $z_{t} = ln{\,(c^{*} - y_{t})}.$ Вважаємо, що $c = c^{*}$ - відоме, тобто, відома асимптота. Отримаємо такий вираз для прогнозного інтервалу:

$$H.M.:z_{n+L} - S_{z}\,k^{*},$$
$$H.M.:z_{n+L} + S_{z}\,k^{*},$$

\noindent де $S_{z}$ - середньоквадратичне віждхилення від тренду $z_{t} = 	ln{\, a} + t\, ln{\,b}.$
Тоді прогнозний інтервал для $y_{n+L}$ має вигляд:

$$H.M.: c^{*} - ant\,ln{(z_{n+L} - S_{z} \, k^{*})},$$
$$B.M.: c^{*} - ant\,ln{(z_{n+L} + S_{z} \, k^{*})},$$

Якщо параметр $c$ невідомий, то можна оцінити параметри модифікованної експоненти $y = c + a \, b^{t}$ методом трьох сум. Згідно з цим методом часовий ряд розбиваємо на три однакових відрізка і позначимо через $\sum_{1}y_{t}, \sum_{2}y_{t}, \sum_{3}y_{t}$ суми рівнів кожного з відрізків, $n$ -кількість рівнів у кожному з відрізків. Вікростовуючи алгоритм методу трьох сум отримаємо такі оцінки параметрів $c$, $a$ і $b$:

$$ b = \sqrt[n]{\frac{\sum_{3}y_{t} - \sum_{2}y_{t}}{\sum_{2}y_{t} - \sum_{1}y_{t}}},\; a = ( \sum\nolimits_{2} y_{t} - \sum\nolimits_{1} y_{t} ) \frac{b - q}{(b^{n} - 1)^2},$$
$$ c = \frac{1}{n} \left[ \frac{\sum_{1}y_{t} \cdot \sum_{3}y_{t} - (\sum_{2}y_{t})^2}{\sum_{1} y_{t} + \sum_{3} y_{t} - 2 \sum_{2}y_{t}} \right].$$

Зауважимо, що метод трьох сум працює, якщо коливанняряду досить малі і результати не дуже чутливі до похибок. Тому перед оцінкою ряд необхідно згладити за допомогою ковзної середньої, якщо у ряді досить сильні коливання, або усунути досить великі викиди і замінити їх на усереднені.
\end{document}